\documentclass[11pt]{scrartcl}
\newcommand{\skipline}{\vskip 0.2in}

\usepackage[sexy]{C:/Users/brian/Documents/OTIS/evan}

\title{Art School}
\author{Brian Zhang}
\date{DGX}

\begin{document}

\maketitle

\section{Shortlist 2019 G1}
Let $T'$ be the point on $(ADE)$ such that $AT' \parallel BC$. Then, we have \[\dang TFD = \dang TAD = \dang DBF,\] hence $AT'$ is tangent to $(BDF)$ as desired.

\section{IMO 2020/1}
The 3 lines meet at the circumecenter $O$ of $\triangle PAB$. Note that $ADPO$ is cyclic, as \[\angle AOP = 2 \cdot \angle PBA = \angle PAD + \angle DPA.\] Then, $OA=OP$ and $OB=OP$ as desired.

\section{APMO 2018/1}
\begin{claim}
$FM=FN$.
\end{claim}
\begin{proof}
Note that \[\dang FMN = \dang FKL = \dang MBH = 90 - \dang A = \dang NCH = \dang NLH = \dang FNM,\]
as desired. This also implies that $F$ lies on the radical axis of $(BMH)$ and $(CNH)$, or $FH$ is tangent to the two circles.
\end{proof}

\begin{claim}
$J$ lies on $FH$.
\end{claim}
\begin{proof}
Note that \[\dang FHM = \dang MKH = \dang NCH = \dang FHN,\] thus $FH$ bisects $\angle MHN$ which gives us what we want.
\end{proof}

\begin{claim}
$AMJN$ is cyclic.
\end{claim}
\begin{proof}
Note that \[\angle MJN = 90 + \frac12 \angle MHN = 90 + \angle MHJ = 90 + \angle MBH = 90 + 90 - \angle A = 180-\angle A\] as desired.
\end{proof}

\begin{claim}
$F$ is the circumcenter of $(AMJN)$.
\end{claim}
\begin{proof}
It suffices to show that $FM=FJ$. Note that
\begin{align*}
\angle FJM &= 180 - \angle MJH = 90 - \frac 12 \angle MNH \\
\angle FMJ &= \frac 12 \angle HMN + \angle FMN = \frac 12 \angle HMN + \frac 12 \angle MHN
\end{align*}
It is easy to see that these two are equal, so we are done.
\end{proof}

\section{USAMO 2021/1}
Let $(ABB_1A_2)$ and $(CAA_1C_2)$ intersect at $P$. Then, the angle condition implies that $\angle BPC + \angle BC_1C = 180$, hence $(BCC_1B_2)$ hits $P$. To finish, note that $APB_2$ is collinear, since $\angle AP_1C = \angle B_2PC = 90^\circ$, thus $B_1C_2, C_1A_2, A_1B_2$ concur at the point $P$. 

\end{document}
